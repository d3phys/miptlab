\documentclass[12pt,a4paper]{article}
\usepackage[T2A]{fontenc}
\usepackage[utf8]{inputenc}
\usepackage[russian]{babel}
\usepackage{amsmath}
\usepackage{amssymb}
\usepackage{graphicx}
\usepackage{floatrow}
\usepackage{booktabs}
\usepackage{wrapfig}
\usepackage{lipsum}
\usepackage{subcaption}


\newcommand{\figref}[1]{(См. рис. \ref{#1})}
\newcommand{\secref}[1]{(См. раздел. \ref{#1})}

\newcommand{\e}[1]{\text{$\cdot10^{#1}$}}



\author{\normalsize Выполнил: Дедков Денис, группа Б01-109 \\
	\normalsize 28.02.2022}
\date{}



\usepackage{float}
\restylefloat{table}
\title{
	\large Отчет о выполнении лабораторной работы 2.4.1 \\
	\Large Определение теплоты испарения жидкости \\ 
	
}


\begin{document}
	\maketitle
	\subsection*{Цель работы} 
	  1) измерение объемов форвакуумной и высоковакуумной частей установки; 2) определение скорости откачки системы в стационарном режиме, а также по ухудшению и по улучшению вакуума. 
	
	\subsection*{Оборудование и приборы} Экспериментальный стенд на основе компактного безмасляного высоковакуумного откачного поста Pfeiffer Vacuum серии
	HiCube 80 Eco с диафрагменным и турбомолекулярным насосами, вакуумметров Pfeiffer Vacuum серии DigiLine, и вакуумных быстроразъёмных
	компонентов. Блок управления (цифровой интерфейс RS-485).
	
	
\subsection*{Теоретическое введение}



\subsection*{Экспериментальная установка}



\subsection*{Ход работы}





\subsection*{Вывод}


Из отчета видно, что на точность полученных результатов (см. таблицу \ref{tab:itog}) влияет не только 
 



\end{document}