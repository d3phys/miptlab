\documentclass[12pt,a4paper]{article}
\usepackage[T2A]{fontenc}
\usepackage[utf8]{inputenc}
\usepackage[russian]{babel}
\usepackage{amsmath}
\usepackage{amssymb}
\usepackage{graphicx}
\usepackage{floatrow}

\newcommand{\figref}[1]{(См. рис. \ref{#1})}
\newcommand{\secref}[1]{(См. раздел. \ref{#1})}

\newcommand{\e}[1]{\text{$\cdot10^{#1}$}}

\title{
	\large Отчет о выполнении лабораторной работы 2.1.4 \\
	\Large Определение теплоемкости твердых тел \\ 

}

\author{\normalsize Выполнил: Дедков Денис, группа Б01-109 \\
	\normalsize 06.02.2022}
\date{}
\begin{document}
	\maketitle
	
	\textbf{Цель работы:} измерение количества подведенного тепла и вызванного им нагрева твердого тела; определение теплоемкости по экстраполяции отношения $\Delta Q / \Delta T$ к нулевым потерям тепла.
	
	\textbf{В работе используются:} калориметр с нагревателем и термометром сопротивления; амперметр; вольтметр; мост постоянного тока; источник питания 36 В.
	
	\subsubsection*{Теоретическое введение}
	
	В данной работе теплоемкость определяется по формуле
	\begin{equation}
		C = \frac{\Delta Q}{\Delta T},
		\label{eq:dQdT}
	\end{equation}
	
	где $\Delta Q$ -- количество тепла, подведенного к телу, и $\Delta T$ -- изменение температуры тела, произошедшее в результате подвода тепла.
	
	Температура исследуемого тела надежно измеряется термометром сопротивления, а определение количества тепла, поглощенного телом, обычно вызывает затруднение. В реальных условиях не вся энергия $P \Delta t$, выделенная нагревателем, идет на нагревание исследуемого тела и калориметра, часть ее уходит из калориметра благодаря теплопроводности его стенок. Оставшееся в калориметре количество тепла $\Delta Q$ равно 
	\begin{equation}
		\Delta Q = P\Delta t - \lambda(T - T_{\text{к}}) \Delta t,
		\label{eq:dQ}
	\end{equation}
	где $P$ -- мощность нагревателя, $\lambda$ -- коэффициент теплоотдачи стенок, $T$ -- температура тела, $T_{\text{к}}$ -- комнатная температура, $ \Delta t$ -- время, в течение которого идет нагревание.
	
	Из уравнений (1) и (2) получаем
	\begin{equation}
		C = \frac{P - \lambda(T - T_{\text{к}})}{\Delta T / \Delta t}
		\label{osnovnaya}
	\end{equation}
	Формула (3) является основной расчетной формулой. Она определяет теплоемкость тела вместе с калориметром. Теплоемкость калориметра измеряется отдельно и вычитается из результата.
	
	С увеличением температуры исследуемого тела растет утечка энергии, связанная с теплопроводностью стенок калориметра. Из формулы (2) видно, что при постоянной мощности нагревателя по мере роста температуры количество тепла, передаваемое телу, уменьшается, и, следовательно, понижается скорость изменения его температуры.
	
	Погрешности, связанные с утечкой тепла, оказываются небольшими, если не давать телу заметных перегревов и проводить все измерения при температурах, мало отличающихся от комнатной. Однако при небольших перегревах возникает большая ошибка при измерении $\Delta T = T - T_\text{к}$, и точность определения теплоемкости не возрастает. Чтобы избежать этой трудности, в работе используется следующая методика измерений. Зависимость скорости нагревания тела $\Delta T / \Delta t$ от температуры измеряется в широком интервале изменения температур. По полученным данным строится график
	\begin{equation*}
		\frac{\Delta T}{\Delta t} = f(T).
	\end{equation*}
	Этот график экстраполируется к температуре $T = T_{\text{к}}$, и таким образом определяется скорость нагревания при комнатной температуре $(\Delta T / \Delta t)_{T_{\text{к}}}$. Подставляя полученное выражение в формулу (3) и замечая, что при $T = T_{\text{к}}$ член $\lambda(T - T_{\text{к}})$ обращается в ноль, получаем
	\begin{equation}
		C = \frac{P}{(\Delta T / \Delta t)_{T_{\text{к}}}}
		\label{4}
	\end{equation}

	
\begin{figure}
	\centering
	\includegraphics[width=0.7\linewidth]{Калориметр}
	\caption{Схема устройства калориметра}
	\label{fig:}
\end{figure}
	
	Температура измеряется термометром сопротивления, который представляет собой медную проволоку, намотанную на теплопроводящий каркас внутренней стенки калориметра (рис. 1). Сопротивление проводника изменяется с температурой по закону
	
	\begin{equation}
		R_{T} = R_{0}(1 + \alpha \Delta T),
		\label{RT}
	\end{equation}
	
	где $R_{T}$ -- сопротивление термометра про $T  ^{\circ}C$, $R_{0}$ -- его сопротивление при $0  ^{\circ}C$, $\alpha$ -- температурный коэффициент сопротивления. 
	
	Дифференцируя (5) по времени, найдем
	
	\begin{equation}
		\frac{dR}{dt} = R_{0}\alpha \frac{dT}{dt},
		\label{dRT}
	\end{equation}
	
	Выразим сопротивление $R_{0}$ через измеренное значение $R_{\text{к}}$ -- сопротивление термометра при комнатной температуре. Согласно (5), имеем
	
	\begin{equation}
		R_{0} = \frac{R_{\text{к}}}{1 + \alpha \Delta T_{\text{к}}},
		\label{R0}
	\end{equation}
	
	Подставляя (6) и (7) в (4), найдем
	
	\begin{equation}
		C = \frac{PR_{\text{к}} \alpha}{(\frac{dR}{dt})_{T_{\text{к}}}(1 + \alpha \Delta T_{\text{к}})},
		\label{capacity}
	\end{equation}
	
	Входящий в формулу температурный коэффициент сопротивления меди равен $\alpha = 4,28 \cdot 10^{-3}~\text{град}^{-1}$, все остальные величины определяются экспериментально. 
	
	\subsubsection*{Экспериментальная установка}
		\begin{figure}[h!]
		\centering
		\includegraphics[width=0.4\textwidth]{"Схема нагревателя"}
		\caption{Схема включения нагревателя}
		\label{fig:heater}
	\end{figure}
	Установка состоит из калориметра с пенопластовой изоляцией, помещенного в ящик из многослойной клееной фанеры. Внутренние стенки калориметра выполненным из материала с высокой теплопроводностью. Надежность теплового контакта между телом и стенками обеспечивается их формой: они имеют вид усеченных конусов и плотно прилегают друг к другу. В стенку калориметра вмонтированы электронагреватель и термометр сопротивления. Схема включения нагревателя изображена ниже \figref{fig:heater}.
		


 Система реостатов позволяет установить нужную силу тока в цепи нагревателя. По амперметру и вольтметру определяется мощность, выделяемая в нагревателе. Величина сопротивления термометра измеряется мостом постоянного тока.


Определим параметры установки: 
$$R_k = 18.18 \pm 0.1 \text{ Ом},$$
$$P = 10.8 \text{ Вт},$$
$$\alpha = 4.28\cdot10^{-3} \text{ K}^{-3}.$$
Массы образцов запишем в таблицу \ref{tab1:mat}:
\begin{table}[h!]
	\label{tab1:mat}
	\centering
	\footnotesize
	\begin{tabular}{|c|c|c|c|}
		\hline
		Материал образца: & Железо          & Латунь           \\ \hline
		Масса образца, г  & $815,1 \pm 0,1$ & $875,5 \pm 0,1$  \\ \hline
	\end{tabular}
	\caption{Параметры исследуемых образцов}
	\label{tab:param_of_facility}
\end{table}

\subsubsection*{Ход работы}
Снимем зависимость R(t) для калориметра, а также для 2 исследуемых образцов. Данные занесем в таблицу \ref{tab:data}.

\begin{table}
	\label{tab:data}
	\caption{Таблица $R(t)$}
	\centering
	\footnotesize
	\begin{tabular}{|r|r|r|r|r|r|}
		\hline 
		\multicolumn{2}{c}{Калориметр} & \multicolumn{2}{c}{Латунь} & \multicolumn{2}{c}{Железо} \\ \hline 
		
		t, с &  R, \text{ кОм}  &    t, с &  R, \text{ кОм} &    t, с &  R, \text{ кОм} \\ \hline
		0.00 & 18.177 &   0.00 & 18.325 &    0.00 & 18.425 \\ \hline
		30.86 & 18.225 &  43.42 & 18.375 &   62.22 & 18.475 \\ \hline
		73.96 & 18.275 &  92.94 & 18.425 &  121.00 & 18.525 \\ \hline
		118.58 & 18.325 & 145.26 & 18.475 &  185.01 & 18.575 \\ \hline
		166.39 & 18.375 & 202.18 & 18.525 &  253.12 & 18.625 \\ \hline
		214.05 & 18.425 & 262.96 & 18.575 &  324.68 & 18.675 \\ \hline
		263.97 & 18.475 & 320.04 & 18.625 &  401.12 & 18.725 \\ \hline
		314.49 & 18.525 & 387.82 & 18.675 &  482.57 & 18.775 \\ \hline
		367.38 & 18.575 & 455.25 & 18.725 &  564.47 & 18.825 \\ \hline
		422.03 & 18.625 & 531.25 & 18.775 &  655.75 & 18.875 \\ \hline
		478.91 & 18.675 & 606.94 & 18.825 &  746.58 & 18.925 \\ \hline
		536.51 & 18.725 & 690.19 & 18.875 &  859.31 & 18.975 \\ \hline
		596.39 & 18.775 & 749.88 & 18.925 &  943.78 & 19.025 \\ \hline
		658.04 & 18.825 & 840.60& 18.975 & 1044.53 & 19.075 \\ \hline
		720.93 & 18.875 &      &     & 1149.25 & 19.125 \\ \hline
		785.14 & 18.925 &      &     & 1210.19 & 19.175 \\ \hline
		851.05 & 18.975 &      &     &       &  \\ \hline
	\end{tabular}
\end{table}


Построим графики измеренных зависимостей $R_T = R(t)$:  
\begin{figure}
	\centering
	\includegraphics[width=0.9\linewidth]{"Сопротивление от температуры"}
	\caption{Сопротивление от температуры}
	\label{fig:RotT}
\end{figure}


Основной метод данной работы - численный расчет производных. Рассчитаем производную по разности соседних измерений (таблицы \ref{tab:pkal}, \ref{tab:piron}, \ref{tab:plat}):

$$\frac{dR}{dt} \approx \frac{R(t_2) - R(t_1)}{t_2-t_1}.$$

Исходно измеренные зависимости $R(t)$ напоминают полином небольшой степени. Следовательно можно попробовать экстраполировать производные квадратичной зависимостью. В самом деле, глядя на графики $\frac{dR}{dt}(R)$, данная модель неплохо описывает зависимости.
Также для обоснования линейной модели можно заметить, 

Для обработки будет использована разновидность метода $\chi^2$ - метод наименьших квадратов.


 
\begin{table}
	\centering
	\caption{Статистическая обработка}
	\label{tab:stat}
	\footnotesize
	\begin{tabular}{|c|c|c|}
		
		\hline
		Система: & $\chi^2$ & Критерий $\chi^2$ \\ \hline
		Калориметр & 2.2 & 0.2  \\ \hline
		Железо 	   & 0.7 & 0.05  \\ \hline
		Латунь 	   & 1.5 & 0.3   \\ \hline
		
	\end{tabular}
	
\end{table} 

\begin{table}
	\centering
	\caption{Коэффициенты}
	\label{tab:coeff}
	\footnotesize
	\begin{tabular}{|c|c|c|c|c|c|c|}
		
		\hline
		Система: & $\lambda_0$ & $\sigma_{\lambda_0}$ &$\lambda_1$ & $\sigma_{\lambda_1}$ &$\lambda_2$ & $\sigma_{\lambda_2}$ \\ \hline
		Калориметр 	& 0.00031244 & 4.1e-08 & -0.01218001 & 7.6e-07 & 0.1193555 & 1.4e-05\\ \hline
		Железо      & 0.00055448 & 4.4e-08 & -0.02136197 & 8.2e-07 & 0.2062121 & 1.5e-05\\ \hline
		Латунь 	   & 0.00105938 & 4.6e-08 & -0.04031611  & 8.5e-07 & 0.38413587 & 1.6e-05\\ \hline
	\end{tabular}

\end{table} 
 
 
 \begin{table}
 	\centering
 	\caption{Зависимости}
 	\label{tab:lin}
 	\footnotesize
 	\begin{tabular}{|c|c|}
 		
 		\hline
 		Система:   & Зависимость  \\ \hline
 		Калориметр &  $y = 0.00031x^2 -0.012x+  0.12$ \\ \hline
 		Железо     &  $y = 0.00055x^2 -0.021x + 0.21$ \\ \hline
 		Латунь     &  $y = 0.0011x^2 -0.040x+ 0.38$ \\ \hline
 		
 	\end{tabular}
 	
 \end{table}


Перейдем к расчету зависимостей. Сделаем это отдельно для каждого образца. Для квадратичной зависимости используем следующее выражение:

$$f(x) = \lambda_0x^2 + \lambda_1x + \lambda_2,$$
Обозначим вектор параметров за $\lambda = (\lambda_0, \lambda_1, \lambda_2)$.
Для определения коэффициентов минимизируем квадрат отклонения, а именно следующую функцию:
$$\chi^2(\lambda) \rightarrow \text{min}.$$
$$\chi^2(\lambda) = \sum_k{\left(\frac{y_k - f(x_k| \lambda)}{\sigma_y}\right)^2} \approx \frac{1}{\sigma^2}\sum_k{\left(y_k - f(x_k | \lambda)\right)^2}.$$

Погрешность отдельного расчета производной примем $\sigma \approx 5\e{-5}$.
Для качественного критерия применимости модели будем использовать следующую величину:
$$\Theta = \frac{\chi^2}{n - p},$$
где $n$ - кол-во измерений,
	$p$ - число степеней свободы.
	
При хорошем соответствии эксперимента и теории $\Theta \backsim 1$.	
	
Из теории методов обработки следует, что для оценки погрешности параметров, нужно решить уравнение 
$$\chi^2(\lambda + \sigma_{\lambda}) - \chi^2(\lambda) = 1$$.
Откуда получается выражение для ошибки определения параметра.

Для оценки точности аппроксимации проведем в соответствии с правилами для погрешности косвенных измерений возьмем: 
$$\Delta y \approx \sqrt{\left(x_0^2\Delta\lambda_0\right)^2+\left(x_0\Delta\lambda_1\right)^2+\left(\Delta\lambda_2\right)^2}$$
\begin{figure}\CenterFloatBoxes
	\begin{floatrow}
		\ffigbox[\FBwidth]
		{\caption{Производные для пустого калориметра}\label{fig:pkal}}
		{
			
			\includegraphics[width=1.15\linewidth]{"Производные калориметр.pdf"}
			
		}
		\killfloatstyle\ttabbox[\Xhsize]
		{\caption{Производные для пустого калориметра} \label{tab:pkal}}
		{	\footnotesize
			\begin{tabular}{|c|c|}
				
				\hline
				$\frac{\partial R}{\partial t}$, $\frac{\text{Ом}}{\text{c}}$ &        $R$ \text{Ом}\\ \hline
				0.001555 & 18.177 \\ \hline
				0.001160 & 18.225 \\ \hline
				0.001121 & 18.275 \\ \hline
				0.001046 & 18.325 \\ \hline
				0.001049 & 18.375 \\ \hline
				0.001002 & 18.425 \\ \hline
				0.000990 & 18.475 \\ \hline
				0.000945 & 18.525 \\ \hline
				0.000915 & 18.575 \\ \hline
				0.000879 & 18.625 \\ \hline
				0.000868 & 18.675 \\ \hline
				0.000835 & 18.725 \\ \hline
				0.000811 & 18.775 \\ \hline
				0.000795 & 18.825 \\ \hline
				0.000779 & 18.875 \\ \hline
				0.000759 & 18.925 \\ \hline
			\end{tabular}
			
		}
	\end{floatrow}
\end{figure}




\begin{figure}\CenterFloatBoxes
	\begin{floatrow}
		\ffigbox[\FBwidth]
		{\caption{Производные для железного образца}\label{fig:piron}}
		{
			
			\includegraphics[width=1.15\linewidth]{"Производные железо.pdf"}
			
		}
		\killfloatstyle\ttabbox[\Xhsize]
		{\caption{Производные для железного образца} \label{tab:piron}}
		{	\footnotesize
			\begin{tabular}{|c|c|}
				
				\hline
				$\frac{\partial R}{\partial t}$, $\frac{\text{Ом}}{\text{c}}$ &        $R$ \text{Ом}\\ \hline
				0.000804 & 18.425 \\ \hline
				0.000851 & 18.475 \\ \hline
				0.000781 & 18.525 \\ \hline
				0.000734 & 18.575 \\ \hline
				0.000699 & 18.625 \\ \hline
				0.000654 & 18.675 \\ \hline
				0.000614 & 18.725 \\ \hline
				0.000611 & 18.775 \\ \hline
				0.000548 & 18.825 \\ \hline
				0.000551 & 18.875 \\ \hline
				0.000444 & 18.925 \\ \hline
				0.000592 & 18.975 \\ \hline
				0.000496 & 19.025 \\ \hline
				0.000477 & 19.075 \\ \hline
				0.000820 & 19.125 \\ \hline
			\end{tabular}
			
		}
	\end{floatrow}
\end{figure}

\begin{figure}\CenterFloatBoxes
	\begin{floatrow}
		\ffigbox[\FBwidth]
		{\caption{Производные для латунного образца}\label{fig:plat}}
		{
			
			\includegraphics[width=1.15\linewidth]{"Производные латунь.pdf"}
			
		}
		\killfloatstyle\ttabbox[\Xhsize]
		{\caption{Производные для латунного образца} \label{tab:plat}}
		{	\footnotesize
			\begin{tabular}{|c|c|}
				
				\hline
				$\frac{\partial R}{\partial t}$, $\frac{\text{Ом}}{\text{c}}$ &        $R$ \text{Ом}\\ \hline
				0.001152 & 18.325 \\ \hline
				0.001010 & 18.375\\ \hline
				0.000956 & 18.425 \\ \hline
				0.000879 & 18.475 \\ \hline
				0.000823 & 18.525 \\ \hline
				0.000876 & 18.575 \\ \hline
				0.000738 & 18.625 \\ \hline
				0.000742 & 18.675 \\ \hline
				0.000658 & 18.725 \\ \hline
				0.000661 & 18.775 \\ \hline
				0.000601 & 18.825 \\ \hline
				0.000838 & 18.875 \\ \hline
				0.000551 & 18.925 \\ \hline
			\end{tabular}
			
		}
	\end{floatrow}
\end{figure}
Теперь находим $\left(\frac{dR}{dt}\right)_{T_k}$ относительно $x$ для каждого образца. 
Откуда получаем оценки значений для $\left(\frac{dR}{dt}\right)_{T_k}$.  Занесем результаты в таблицу \ref{tab:capacity}.


Понятно, что ключевую роль в погрешности определения теплоемкости будет иметь именно экстраполяция данных.
Так как результат представляет собой произведение независимых величин, погрешность теплоемкости можно оценить как сумму квадратов относительных погрешностей. Учитывая то, что погрешность экстраполяции намного превосходит погрешности остальных величин, можно записать:
$$\varepsilon_C \approx \varepsilon_{\left(\frac{dR}{dt}\right)_{T_k}}.$$
Затем используем формулу \ref{capacity}.


\begin{table}
	\centering
	\caption{Результаты расчета теплоемкостей}
	\label{tab:capacity}
	\footnotesize
	\begin{tabular}{|c|c|c|c|c|c|c|}
		
		\hline
		Система:   &$\left(\frac{dR}{dt}\right)_{T_k}$, $\frac{\text{кОм}}{\text{с}}$ & $\Delta \left(\frac{dR}{dt}\right)_{T_k}$, $\frac{\text{кОм}}{\text{с}}$ &  $C_1$, $\frac{\text{Дж}}{\text{К}}$  & $C_1-C_0$, $\frac{\text{Дж}}{\text{К}}$ & $c$, $\frac{\text{Дж}}{\text{К}\cdot \text{кг}}$ & $\varepsilon_{c} $				  \\ \hline
		Калориметр &  0.00119 & 2\e{-5} & 649 & -  & - & 0.04  \\ \hline
		Железо     &  0.000807 & 3\e{-5} & 983 & 333 & 408 & $0.05$ \\ \hline
		Латунь     &  0.000890 & 3.5\e{-3} & 883 & 235 & 270 & $0.05$ \\ \hline
		
	\end{tabular}
	
	
	
\end{table}

\clearpage

\subsubsection*{Вывод}
Получены довольно интересные результаты. 

Основная ошибка в определении теплоемкостей твердых тел связана с разбросом точек и численного расчета производных, и как следствие  ошибки экстраполяции данных. 
$$c_{\text{железо}} \approx (4.1 \pm 0.2)\e{2} \frac{\text{Дж}}{\text{К}\cdot \text{кг}}$$
$$c_{\text{латунь}} \approx (2.70 \pm 0.15)\e{2} \frac{\text{Дж}}{\text{К}\cdot \text{кг}}$$
Основная причина полученных результатов - недостаток эксперимента. Легко обнаружить, что именно первый проведенный эксперимент привел к наиболее точным результатам. А значит в последующих экспериментах тело просто не успевало остыть, а система установить равновесие. 

Однако метод определения теплоемкостей можно считать достаточно точным. О чем свидетельствуют адекватные значения критерия $\chi^2$ (см. таблицу \ref{tab:stat})
 

\end{document}